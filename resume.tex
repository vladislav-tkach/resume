%%%%%%%%%%%%%%%%%%%%%%%%%%%%%%%%%%%%%%%%%
% Medium Length Professional CV
% LaTeX Template
% Version 3.0 (December 17, 2022)
%
% This template originates from:
% https://www.LaTeXTemplates.com
%
% Author:
% Vel (vel@latextemplates.com)
%
% Original author:
% Trey Hunner (http://www.treyhunner.com/)
%
% License:
% CC BY-NC-SA 4.0 (https://creativecommons.org/licenses/by-nc-sa/4.0/)
%
%%%%%%%%%%%%%%%%%%%%%%%%%%%%%%%%%%%%%%%%%

%----------------------------------------------------------------------------------------
%	PACKAGES AND OTHER DOCUMENT CONFIGURATIONS
%----------------------------------------------------------------------------------------

\documentclass[
	%a4paper, % Uncomment for A4 paper size (default is US letter)
	11pt, % Default font size, can use 10pt, 11pt or 12pt
]{resume} % Use the resume class

\usepackage{fontawesome} % Font containing web-related icons
\usepackage{hyperref} % Extensive support for hypertext
\usepackage{xcolor} % Driver-independent color extensions

%------------------------------------------------

\name{Vladyslav Tkach} % Your name to appear at the top

% You can use the \address command up to 3 times for 3 different addresses or pieces of contact information
% Any new lines (\\) you use in the \address commands will be converted to symbols, so each address will appear as a single line.

%\address{123 Broadway \\ City, State 12345} % Main address

%\address{123 Pleasant Lane \\ City, State 12345} % A secondary address (optional)

%\address{(011)~$\cdot$~899~$\cdot$~9881 \\ john@LaTeXTemplates.com} % Contact information

%----------------------------------------------------------------------------------------

\begin{document}

\begin{center}

    {\Large \underline{C++ Software Engineer}} % Your position
    \medskip

\end{center}

\begin{center}

    \faPhone{ +38 (099) 115 9923 } % Your phone number
    \hfill
    \faEnvelope{ vladislautkach@gmail.com } % Your email
    \hfill
    \faPaperPlane{ \href{https://t.me/vladyslav_tkach_tg}{\color{blue}\underline{@vladyslav\_tkach\_tg}} } % Your Telegram
    \hfill
    \faLinkedin{ \href{https://www.linkedin.com/in/vladyslav-tkach/}{\color{blue}\underline{vladyslav-tkach}} } % Your LinkedIn

\end{center}

%----------------------------------------------------------------------------------------
%	TECHNICAL STRENGTHS SECTION
%----------------------------------------------------------------------------------------

\begin{rSection}{Technical Strengths}

    \begin{center}
    
        \centering
        \begin{tabular}{
            @{}
            >{\bfseries}l @{\hspace{3ex}}|
            @{\hspace{3ex}} l @{\hspace{3ex}}|
            @{\hspace{3ex}} l
        }
             & \textbf{Experienced at} & \textbf{Had experience with} \\
            \hline
		  Programming Languages & C++, Python & C, C\#, Java \\
            \hline
		  Development Tools & CMake, Git & GiHub Actions, GNU Make, gdb \\
            \hline
		  Technologies & Qt framework, GoogleTest & GNU/Linux, Boost, OpenCV \\
            \hline
		  Spoken Languages & \multicolumn{2}{@{} l}{Native Ukrainian speaker, advanced English proficiency} \\
	   \end{tabular}

    \end{center}

\end{rSection}

%----------------------------------------------------------------------------------------
%	WORK EXPERIENCE SECTION
%----------------------------------------------------------------------------------------

\begin{rSection}{Experience}

	\begin{rSubsection}{\href{https://www.materialise.com/en}{\color{blue}Materialise} - \href{https://www.materialise.com/ja/healthcare/mimics-enlight-cmf}{\color{blue}Mimics Enlight CMF}}{October 2021 - Present}{Software Engineer (C++, Python)}{Kyiv, Ukraine}
        \item Developed, tested, debugged, and documented a planner application for cranio-maxillofacial surgical operations using C++ and Qt framework.
        \item Strengthened Mimics Enlight cross-product platform and engaged in refactoring of existing code.
		\item Applied the MatSDK libraries collection for implementation of new APIs to improve both performance and UX.
        \item Worked within an agile team following Scrum methodology, actively participated in cross-team communication and collaboration.
        \item Had experience with creating scripted UI tests using Mimics Python API and proprietary testing framework.
	\end{rSubsection}

%------------------------------------------------

    \begin{rSubsection}{\href{https://www.materialise.com/en}{\color{blue}Materialise} - \href{https://www.materialise.com/en/careers/students-graduates/internships}{\color{blue}C++ Academy}}{July 2021 - September 2021}{Intern (C++)}{Kyiv, Ukraine}
        \item Worked on a 3D printing project in a team following Scrum methodology.
        \item Implemented 3D geometry algorithms and used Qt3D for custom rendering and camera controls.
        \item Learned about and practiced C++17 thoroughly including hands-on STL.
        \item Investigated OOP concepts as well as design patterns with exercises.
        \item Studied high-level debugging and error handling, test-driven development(TDD) technique, profiling and disassembly as well as parallel programming.
        \item Learned Qt framework, practiced both Qt Widgets and QML.
    \end{rSubsection}

\end{rSection}

%----------------------------------------------------------------------------------------
%	EDUCATION SECTION
%----------------------------------------------------------------------------------------

\begin{rSection}{Education}

    \href{http://csc.knu.ua/en/}{\color{blue}\textbf{Taras Shevchenko National University of Kyiv}} \hfill {September 2019 - June 2023} \\ 
    \textit{Faculty of Cybernetics and Computer Science} \hfill \textit{Kyiv, Ukraine} \\ 
    B.S. in Computer Science \\

    \href{http://upml.knu.ua/}{\color{blue}\textbf{Ukrainian Physics and Mathematics Lyceum of Kyiv University}} \hfill {September 2016 - May 2019} \\ 
    \textit{GPA 10.5} \hfill \textit{Kyiv, Ukraine}
\end{rSection}

%----------------------------------------------------------------------------------------
%	EXAMPLE SECTION
%----------------------------------------------------------------------------------------

%\begin{rSection}{Section Name}

	%Section content\ldots

%\end{rSection}

%----------------------------------------------------------------------------------------

\end{document}
