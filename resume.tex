\documentclass[11pt]{resume}

\usepackage{fontawesome}
\usepackage{hyperref}
\usepackage{xcolor}

\name{Vladyslav Tkach}
\specialty{Software Engineer}

\begin{document}
    \hspace*{\fill}
    \faEnvelope{ }{vladislautkach@gmail.com}
    \hspace*{\fill}
    \faPaperPlane{ }{\href{https://t.me/vladyslav_tkach_tg}{\color{blue}\underline{\smash{@vladyslav\_tkach\_tg}}}}
    \hspace*{\fill}
    \faLinkedin{ }{\href{https://www.linkedin.com/in/vladyslav-tkach/}{\color{blue}\underline{\smash{vladyslav-tkach}}}}
    \hspace*{\fill}

    \begin{rSection}{Experience}
    	\begin{rSubsection}
            {\href{https://www.materialise.com/en}{\color{blue}\underline{\smash{Materialise}}} - \href{https://www.materialise.com/en/healthcare/mimics-enlight-cmf}{\color{blue}\underline{\smash{Mimics Enlight CMF}}}}{October 2021 - Present}{Software Engineer (C++, Python)}{Kyiv, Ukraine}
            
            \item Developed, tested, debugged, and documented a medical planner application for craniomaxillofacial surgical operations using \textbf{C++ and Qt framework}.
            \item Strengthened Mimics Enlight \textbf{cross-product platform} and engaged in \textbf{refactoring} of existing code.
    		\item Applied the MatSDK library collection implementing \textbf{new APIs} to improve both \textbf{performance and UX}.
            \item Worked within an agile team following \textbf{Scrum} methodology and actively participated in \textbf{cross-team communication} and collaboration.
            \item Facilitated seamless \textbf{onboarding}, providing \textbf{mentorship} to new team members, fostering growth and integration.
            \item Had experience with creating scripted UI tests using Mimics \textbf{Python} API and proprietary testing framework.
    	\end{rSubsection}

        \begin{rSubsection}
            {\href{https://www.materialise.com/en}{\color{blue}\underline{\smash{Materialise}}} - \href{https://www.materialise.com/en/careers/students-graduates/internships}{\color{blue}\underline{\smash{C++ Academy}}}}{July 2021 - September 2021}{Intern (C++)}{Kyiv, Ukraine}
            
            \item Worked on a \textbf{3D printing} project in a team following \textbf{Scrum} methodology.
            \item Implemented 3D geometry algorithms and used \textbf{Qt3D} for custom rendering and camera controls.
            \item Learned about and practiced \textbf{C++17} thoroughly including hands-on \textbf{STL}.
            \item Investigated \textbf{OOP} concepts as well as \textbf{design patterns} with exercises.
            \item Studied high-level \textbf{debugging} and error handling, test-driven development(TDD) technique, profiling, and disassembly as well as parallel programming.
        \end{rSubsection}
    \end{rSection}

    \begin{rSection}{Certifications}
        \begin{rSubsection}
            {\href{https://www.uarust.com/}{\color{blue}\underline{\smash{Ukrainian Rust Community}}} - \href{https://github.com/rust-lang-ua/rustcamp}{\color{blue}\underline{\smash{Rust Bootcamp}}}}{December 2023 - April 2024}{Rust, Rust backend}{Ukraine}
            
            \item Covered \textbf{fundamentals} of compiled languages: memory model, typing, polymorphism, ownership, etc.
            \item Examined \textbf{Rust concepts}: memory management (\texttt{Box}, \texttt{Pin}), ownership management (\texttt{Rc, Arc}), dispatching (generics, \texttt{dyn Trait}), thread safety (\texttt{Send}, \texttt{Sync}), interior mutability (\texttt{RefCell}, \texttt{Mutex}), etc.
            \item Practiced \textbf{Rust idioms}: newtype, typestate, memory replace, generic in type out, exhaustivity, sealing, etc.
            \item Applied \textbf{Rust infrastructure}: \texttt{chrono}, \texttt{chumsky}, \texttt{clap}, \texttt{config}, \texttt{im}, \texttt{mockall}, \texttt{proptest}, \texttt{quote}, \texttt{rayon}, \texttt{rust-argon2}, \texttt{serde}, \texttt{sha3}, \texttt{syn}, \texttt{tokio}, \texttt{tracing}, etc.
            \item Discovered \textbf{backend} web-development basics: database integration (PostgreSQL, \texttt{sqlx}, \texttt{sea-query}), HTTP server (\texttt{axum}, \texttt{actix}) and client (\texttt{reqwest}), web API (REST, OpenAPI, GraphQL), auth (JWT, cookies).
        \end{rSubsection}
    \end{rSection}

    \begin{rSection}{Education}
        \begin{rSubsection}
            {\href{http://csc.knu.ua/en/}{\color{blue}\underline{\smash{Taras Shevchenko National University of Kyiv}}}}{September 2019 - June 2023}{Faculty of Cybernetics and Computer Science}{Kyiv, Ukraine}
            
            \item B.S. in Computer Science
        \end{rSubsection}
    \end{rSection}

    \begin{rSection}{Summary}
        Results-oriented software engineer with expertise in Rust, C++, and Python. Proficient in English (C1), native Ukrainian speaker and a dedicated Japanese learner (N4). With a background in medical product development with attention to details and delivering high-quality software solutions.
    \end{rSection}
\end{document}
